\chapter{Introduction}\label{ch:introduction}

Engine overhaul is the process of removing, disassembling, inspecting, repairing, cleaning, reassembling and testing a used engine. Overhaul involves the intricate process of dismantling small parts from the engine—eg. screws, fasteners, nuts and bolts. Those small parts are later reassembled back into a functioning engine.

Thousends of different small parts can be used to structure an engine. To facilitate the process of reassembling the engine, each small part has to be classified and sorted into its position and function after it has been removed.

\subsubsection{Manual Approach}

Traditionally, the tasks of classifying and sorting small parts are executed manually by a human task force. This approach presents a number of problems. Manual sorting is a labour-intensive task that depletes human resources. The manual approach requires the investment of man-hours, which means that it is both time consuming and expensive.

\subsubsection{Automatic Approach}

An automatic small part classification system involves a setup where a camera and a robotic arm are placed over a conveyor belt. First, the small parts are placed on the conveyor belt. Next, the camera takes pictures of the small parts that are rolling underneath on the conveyor belt. The pictures are then sent to an image classification system which in turn classifies the images of the small parts. The classification system sends the labels of the small parts to the robotic arm, which uses those labels to sort the small parts accordingly.

The main brain behind automatic small part classification is the image classification system. In order to build a small part image classification system, we employ convolutional neural networks (CNNs). CNNs have become the state-of-the-art approach to classical computer vision and image processing tasks such as image classification, object detection and many others \cite{krizhevsky2012imagenet} \cite{szegedy2015going}.

The automatic approach involves minimal human interaction, and thus slashes the amount of man power required to sort and classify small parts. Moreover our automatic approach provides a quantifiable measurment for accuracy which can be used to assess and imporve the system, and ultimately reduce the effort required to avoid misclassification errors.


\section{Problem}

Convolutional neural networks are charactaristically data-hungry algorithms. The performance of a CNN algorithm depends on the amount of data that is fed in as input. In our problem's case, we need to use a large number of images of each small part as input to the CNN algorithm. Creating a large number of images for our set of small parts is a time consuming and labour-intensive task. It presents the same problems as the traditional approach of manually labeling small parts.

In this thesis, we propose a system that uses synthetic images to classify small parts. In addition to using real pictures of the small parts, we obtain their respective 3D models and use them to render synthetic images of the real objects. We conduct an experiment to evaluate the performance of using different amounts of synthetic data to train a convolutional neural network.

\section{Motivation}

Most of the cutting edge research in convolutional neural networks has been done using a large number of input data \cite{krizhevsky2012imagenet} \cite{simonyan2014very} \cite{szegedy2015going} \cite{he2016deep}, exploiting large corpora of labeled images \cite{deng2009imagenet}. CNN algorithms are less effective at classifying images if a large amount of input data is absent.

Using synthetic images to classify small objects exploits the power of convolutional neural network while overcoming its need for a large dataset of labeled images. This approach is extendible to image classification problems in other domains. Synthtic images provide a scalable method to generate input for CNN algorithms without the impedence of having to manually create a large dataset.

Furthermore, the process of generating synthetic images gives full control of the surrounding environment to the creator of the synthetic scene. The synthetic scene can easily be adjusted to capture specific features of the small parts.

\section{Objectives}

The main objective of this thesis is to build a system to facilitate the use of synthetic images for small part classification. Furthermore, the system is used to assess the use of synthetic data as input for CNN algorithms to perform image classification. We explore the use of a fully synthetic dataset and also a mixed dataset of both real and synthetic images in varying ratios.

\section{Outline}
In chapter \ref{ch:background}, we describe convolutional neural networks and provide the mathematical background necessary needed to navigate this thesis. We also define the terminology that we use to describe small parts and different types of images in our dataset. Chapter \ref{ch:related_work} transcribes the literature that was reviewed in preparation for this thesis. A review of convolutional neural networks based image classification, image classification in industrial usecases and the usage of syntheic data is provided. In chapter \ref{ch:analysis}, we break down the requirements of our system. We present the system use cases, object model and deployment diagrams. In chapter \ref{ch:system_design}, we describe our subsystems and the relationships between them. We also describe the software and hardware components used for implementation. Chapter \ref{ch:object_design} provides an in-depth explanation of the external frameworks, APIs and software that we use out-of-the-box. Chapter \ref{ch:evaluation} contains our implementation details, experiment design and results. Lastly, chapter \ref{ch:summary} contains a recap of our work, the conclusions we have reached and the potential future work that can be based on our thesis.






