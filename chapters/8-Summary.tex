\chapter{Summary}\label{ch:summary}

In this chapter we recap the work that we have done. We summarize the status of the project by relating to the functional that were previously identified in section \ref{sec:status}. In section \ref{sec:conclusion} we present a conclusion of our results and in section \ref{sec:future} propose a direction for future work that can further advance the usage of synthetic data in image classification.


\section{Status}\label{sec:status}

We look back at the functional requirements that we have defined in chapter \ref{ch:analysis}. We are able to create 3D models of our small parts using the 3DScanner device. However the output 3D models were imprecise and could not capture the details of the small parts. We instead downloaded readily available 3D models. Generating the synthetic image set is implemented using ready made 3D models, the creation of synthetic scenes and rendering 2D synthetic images in Rhinoceros. The transformation ranges for each 3D model are set in the python script that Rhino executes to generate the synthetic images. A camera is used to take real images of the small parts which are then resized by the RealImageProcessor component in the RealImageGenerator device. In the ImageClassifier device, the DatasetSplitter divides the images in the Dataset component into a training set, a validation set and a testing set. The ImageClassifier trains a CNNModel using the training and validation sets, and evaluates its classification accuracy using the testing set.


\section{Conclusion}\label{sec:conclusion}

The main objective of our work was to decrease the amount of manual labor required to label a dataset of small parts. We described a system that facilitates the usage of synthetic data in image classification. Furthermore, we carried out an experiment to evaluate the usage of synthetic images to train a convolutional neural network to classify images of small parts. Our results indicate that a CNN trained on a mixture of synthetic and real images can provide a comparable classification accuracy to CNNs that have been trained on purely real images. This is advantageous in situations where a large dataset of images is not available for new classes.


\section{Future Work}\label{sec:future}

We propose 3 possible directions for future improvements over the work that has been presented in our thesis. Firstly, our results indicate that, when adding new classes, the output classification accuracy is affected by the aesthetic similarity of small parts in the dataset. A logical next step is to evaluate the CNNs after increasing the number of output classes. Secondly, a possible improvement is to evaluate the usage of deeper CNN models such as the ones presented by He et al. \cite{he2016deep} and Szegedy et al. \cite{szegedy2016rethinking}. Thirdly, the total number of synthetic images in the training set has not been varied throughout our experiments. A possible improvement in accuracy could be achieved by training the CNN models on a bigger training set containing a larger amount of synthetic images.
