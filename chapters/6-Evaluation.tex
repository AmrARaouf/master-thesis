\chapter{Evaluation}

In this chapter we describe how we utilized our system to run our experiments. We provide a quantitative analysis of the data that we gathered and the comparison between the different experiment settings.

\section{Objectives}

We hypothesize that we are able to enhance the classification accuracy of our image classifier using synthetic data for training. We examine multiple ratios of synthetic to real images for training. Moreover, we run our experiments on multiple CNN architectures and fine-tune them accordingly.


\section{Methodology}

Our methodology is divided into 2 main parts: Image Collection and Training the Image Classifier. Image Collection is concerned with the real and synthetic images, and how they're collected from small mechanical parts and their corresponding 3D models. Training the Image Classifier revolves around utilizing the collected images to maximize the classification accuracy of the image classifier.

\subsection{Image Collection}

During image collection, we strive to create a synthetic dataset that has a high degree of photo-realism. Simultaniously, we want to create 3D models as fast as possible to decrease the timing of the whole workflow. To achieve this balance, we aim to create an environment that is easy to model on 3D software. We attempt to eliminate light reflections and shadows and reduce the overall complexity of the scene.

\subsubsection{Small Mechanical Parts and their corresponding 3D Models}
We first choose a small mechanical model that we wish for our image classifier to recognize. We then download the corresponding 3D model by searching the Traceparts website \cite{traceparts} for the SMP's code name.

\subsubsection{Collecting Real Images}
We place the chosen small mechanical model on a horizontal plane and place a light source underneath. We choose a plain, white, semi-transparent plane as a background to disperse the light source coming from beneath, and distribute the light evenly accross the plane. The purpose of the dispersed light is to eliminate any shadow that the small mechanical object might cast. Moreover the dispersed back light provides a lighting source without casting a direct light on the small mechanical object, which might cause glare on the SMP's reflective surface.

Next, we place a an iPhone over the plane, such that the small mechanical object is fully within the viewfield of the iPhone's camera. Our hoisting device places the iPhone 133 mm over the plane.

Afterwards, we rotate and change the position of the small mechanical part randomly, while maintaining that the SMP is fully within the viewfield of the camera. We use the iPhone's camera to take a picture of the SMP. We repeat this step until we obtain the desired number of real images.

Lastly, we resize the raw real images taken by the iPhone camera to conform with the height and width required by the image classifier.

\subsubsection{Generating Synthetic Images}
We start by creating the synthetic scene in th Rhinoceros 3D modeling software. First, we take a picture of the real horizontal plane and use it as a background for our synthetic scene. We then place a synthetic lighting source underneath the plane to mimic the lighting effect of the real environment. Next, we place the 3D model on the horizontal plane of the environment.

Next, we generate a python script that uses the Rhinoceros library to manipulate 3D objects in the Rhinoceros software. The python script is the controller for the synthetic data genertor. The synthetic data generator defines the rotation and translation ranges of the 3D model. It also specifies how many images are to be generated. For each new image, the script generates a random rotation and translation value from within the defined ranges. It then applies those transformation to the 3D model and renders a new 2D synthetic image.

\subsubsection{Dataset}


\subsection{Training the Image Classifier}


\section{Results}

\textit{Note: Summarize the most interesting results of your evaluation (without interpretation). Additional results can be put into the appendix.}

\section{Findings}

\textit{Note: Interpret the results and conclude interesting findings}

\section{Discussion}

\textit{Note: Discuss the findings in more detail and also review possible disadvantages that you found}

\section{Limitations}

\textit{Note: Describe limitations and threats to validity of your evaluation, e.g. reliability, generalizability, selection bias, researcher bias}