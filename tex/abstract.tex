\begin{abstract}

Aircraft engines periodically undergo an overhaul process during which the engine parts are dismantled and then later reassembled. When the engine components are taken apart, thousands of small parts are unfastened. To facilitate the step of engine reassembly, the small parts have to be sorted and classified.

In this thesis we present a system to automatically classify small parts using convolutional neural networks. To train the neural networks, we use 3D models of the small parts to render synthetic images. The network is trained using a mixture of images of small parts and synthetic images of their corresponding 3D models. We examine the use of a fully synthetic image set for training, as well as different ratios of real and synthetic images.

Our results indicate that a convolutional neural network trained on a dataset that contains as little as 5\% real images can provide a comparable classification accuracy to networks that have been trained on purely real image sets. This is advantageous in supervised learning problems targeting novel domains, where an annotated dataset of images might not available for training.
\\\\
\textbf{Keywords:} image classification, convolutional neural networks, synthetic images, 3D models

\end{abstract}
