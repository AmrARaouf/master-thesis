\begin{abstract}

\iffalse
Recent advances in Convolutional Neural Networks (CNN) have been able to conquer Computer Vision tasks and even surpass human performance. CNN algorithms are characteristically data-hungry, and obtaining domain-specific labeled data is often a cumbersome manual resource-heavy task. In this text we explore the use of synthetic data to train a CNN to perform image classification of different fasteners.

\textit{Note:}

\textit{\textbf{1. paragraph:} What is the motivation of your thesis? Why is it interesting from a scientific point of view? Which main problem do you like to solve?}

The motivation for my thesis stems from CNN algorithms' hunger for labeled images in order to train for classification tasks and produce performant results. Obtaining a large corpus of labeled images is a difficult task, especially when we wish to train a network to classify a domain-specific dataset.

\textit{\textbf{2. paragraph:} What is the purpose of the document? What is the main content, the main contribution?}
In this document we explore the use of synthetic images to augment the training data for a given CNN. We explore the power of synthetic data to produce acceptable results while using as little real data as possible. The intuition comes from the ease of rendering a large synthetic dataset on demand.

\textit{\textbf{3. paragraph:} What is your methodology? How do you proceed?}
We explore the ratio of synthetic to real images that can be used for training a CNN and obtain good results.
*/
\fi
\end{abstract}