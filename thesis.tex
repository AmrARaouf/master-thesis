\documentclass[a4paper,12pt,twoside]{report}

\usepackage{acronym}
\usepackage{url}
\usepackage{cite}
\usepackage{listings}
\usepackage[pdftex]{graphicx}
\usepackage[hang,small,bf]{caption}
\usepackage{styles/tum}
\usepackage{setspace}
\usepackage[german,english]{babel}
\usepackage{float}
\usepackage{floatflt}
\usepackage{fancyhdr}
\usepackage{color}
\usepackage{booktabs}
\usepackage[pdftex,bookmarks=true,plainpages=false,pdfpagelabels=true]{hyperref}
\usepackage{mdwlist}
\usepackage{enumerate}
\usepackage{paralist}
\usepackage{array}
\usepackage{longtable}
\usepackage{listings}
\usepackage[utf8]{inputenc}
\usepackage[capitalize, noabbrev]{cleveref}

% Path for graphics
\graphicspath{{figures/}}

\begin{document}
\setlength{\evensidemargin}{22pt}
\setlength{\oddsidemargin}{22pt}

\def\doctype{Master's Thesis}
\def\faculty{Informatik}
\def\title{Using Synthetic Data for Classification of Small Parts}		%TODO add title in German
\def\titleGer{Synthetische Daten für die Klassifizierung von Kleinteilen verwenden}	%TODO add title in German
\def\supervisor{Prof. Bernd Brügge, Ph.D.}
\def\advisor{Sajjad Taheri, M.Sc.}
\def\author{Amr Abdelraouf}
\def\date{15.10.2018}		%TODO add submission / handover date


\hypersetup{pdfborder={0 0 0},
                        pdfauthor={Amr Abdelraouf},
                        pdftitle={Using Synthetic Data for Classification of Small Parts},
                        }

\lstset{showspaces=false, numbers=left, frame=single, basicstyle=\small}

\pagenumbering{alph}

\include{tex/cover}
\include{tex/titlepage}
\newpage
\thispagestyle{empty}
\mbox{}
\include{tex/disclaimer}

\newpage
\thispagestyle{empty}
\mbox{}

\chapter*{Acknowledgements}


\pagenumbering{roman}

\selectlanguage{english}
\begin{abstract}

Recent advances in Convolutional Neural Networks (CNN) have been able to conquer Computer Vision tasks and even surpass human performance. CNN algorithms are characteristically data-hungry, and obtaining domain-specific labeled data is often a cumbersome manual resource-heavy task. In this text we explore the use of synthetic data to train a CNN to perform image classification of different fasteners.

\textit{Note:}

\textit{\textbf{1. paragraph:} What is the motivation of your thesis? Why is it interesting from a scientific point of view? Which main problem do you like to solve?}

The motivation for my thesis stems from CNN algorithms' hunger for labeled images in order to train for classification tasks and produce performant results. Obtaining a large corpus of labeled images is a difficult task, especially when we wish to train a network to classify a domain-specific dataset.

\textit{\textbf{2. paragraph:} What is the purpose of the document? What is the main content, the main contribution?}
In this document we explore the use of synthetic images to augment the training data for a given CNN. We explore the power of synthetic data to produce acceptable results while using as little real data as possible. The intuition comes from the ease of rendering a large synthetic dataset on demand.

\textit{\textbf{3. paragraph:} What is your methodology? How do you proceed?}
We explore the ratio of synthetic to real images that can be used for training a CNN and obtain good results.

\end{abstract}

\clearpage

\selectlanguage{german}
\begin{abstract}

%abstract german

\textit{Note: Insert the German translation of the English abstract here.}

\end{abstract}

\clearpage

\selectlanguage{english}


\tableofcontents
\clearpage

\clearpage

\begin{acronym}
\acro{GUI}{Graphical User Interface}

\end{acronym}

\pagenumbering{arabic}

\fancyhead{}
\pagestyle{fancy}
\fancyhead[LE]{\slshape \leftmark}
\fancyhead[RO]{\slshape \rightmark}
\headheight=15pt




%------- chapter 1 -------

\chapter{Introduction}

\textit{Note: Introduce the topic of your thesis, e.g. with a little historical overview.}

\section{Problem}

\textit{Note: Describe the problem that you like to address in your thesis to show the importance of your work. Focus on the negative symptoms of the currently available solution.}

\section{Motivation}

\textit{Note: Motivate scientifically why solving this problem is necessary. What kind of benefits do we have by solving the problem?}

\section{Objectives}

\textit{Note: Describe the research goals and/or research questions and how you address them by summarizing what you want to achieve in your thesis, e.g. developing a system and then evaluating it.}

\section{Outline}

\textit{Note: Describe the outline of your thesis}




%------- chapter 2 -------

\chapter{Background}

\textit{Note: Describe each proven technology / concept shortly that is important to understand your thesis. Point out why it is interesting for your thesis. Make sure to incorporate references to important literature here.}

\section{Supervised Machine Learning}
\subsection{Overview}
\subsection{Image Classification}

\section{Deep Neural Networks}

\section{Convolutional Neural Networks (CNNs)}
\subsection{CNN Building Blocks}
\subsection{CNN Architectures}
\subsubsection{VGG16}
\subsubsection{VGG19}
\subsubsection{Resnet}
\subsubsection{Inception}

\section{Small Mechanical Part Classification}


%------- chapter 3 -------

\chapter{Related Work}

\textit{Note: Describe related work regarding your topic and emphasize your (scientific) contribution in \textbf{contrast} to existing approaches / concepts / workflows. Related work is usually current research by others and you defend yourself against the statement: ``Why is your thesis relevant? The problem was already solved by XYZ.'' If you have multiple related works, use subsections to separate them.}

\section{CNN based Image Classification}
\section{Image Classification using Synthetic Images}


%------- chapter 4 -------

\chapter{Analysis}

%\textit{Note: This chapter follows the Requirements Analysis Document Template in \cite{bruegge2004object}. 
%\textbf{Important:} Make sure that the whole chapter is independent of the chosen technology and development platform. The idea is %that you illustrate concepts, taxonomies and relationships of the application domain independent of the solution domain!
%Cite \cite{bruegge2004object} several times in this chapter.}

\section{Overview}

Our goal is to develop a system that classifies images of different small mechanical parts (SMPs), and to do so as accurately as possible. To realize this goal, we decide to leverage state-of-the-art advancements in Convolutional Neural Network algorithms in the field of computer vision, and more specifically, the sub-field of image classification.

We aim to build a system that scales up well, ie. we would like the classification system to easily adapt new small mechanical parts. Moreover we would like to build a system that can scale up to around 10,000 classes.

CNN algorithms are characteristically data-hungry. The classification accuracy of a CNN algorithm is proportional to the amount of images that are fed in as input. Given the large number of classes, the task of collecting images of each small mechanical part becomes time-consuming and labour-intensive. To combat this problem, we decide to augment the training data of our algorithms with \textit{synthetic images}: 2-dimentional renditions of 3-dimensional computerized graphical models of the small mechanical parts. We hypothesize that our synthetically augmented training set will yield a higher accuracy, whilst minimizing the manual effort needed to take real images of the small mechanical parts.

\section{Requirements}

\subsection{Functional Requirements}

Functional requirements (FRs) describe the interactions between the system and the its environment independent of its implementation. \cite{bruegge2004object}.

The system's main function is to classify images of different small mechanical parts. Furthermore, the system needs to be dynamic and scalable to accomodate new small mechanical parts. To ensure that the system is able to classify images accurately, while leveraging synthetic input images, we introduce the folloing Functional Requirements.

\begin{itemize}
\item [FR1] \textbf{Generate Synthetic Images}: Given a 3D model of a small mechanical part, the system should generate a set of synthetic images of said model. The system should apply a set of random transformations to the 3D model before rendering the 2D synthetic image to ensure that the generated dataset captures the SMP from as many angles in as many positions as possible.

\item [FR2] \textbf{Add New Class to Image Classifier}: The system should accomodate the addition of a new SMP to the classifier. This extends to the ability of the system to accomodate for new input training data, and the ability to label a new output class

\item [FR3] \textbf{Adjust input data split}: The system should allow for changes in the training, validation and testing input data which is assigned as input to the image classifier. This includes changing the number of images in each set and manipulating the ratio of synthetic to real images in the training split.

\item [FR4] \textbf{Retrain Image Classifier}: The system should be dynamic enough to retrain the image classifier. This step is usually taken after the augmenting or adjustment of the input data split.

\item [FR5] \textbf{Fine-Tune Image Classifier}: The underlying CNN algorithm should be subject to fine-tuning in order to maximize the accuracy of the system.
\end{itemize}

\subsection{Nonfunctional Requirements}

Nonfunctional Requirements (NFRs) are key system requirements that apply to the system as a whole. To maintain the system's general requirements we define the following nonfunctional requirements.

\begin{itemize}
\item [NFR1] \textbf{Performance}: A performence requirement is the measure of a quantifiable attribute of our system. In our case we would like to track our system's \textbf{classification accuracy}.\\
We first specify our system's accuracy for a single class to be the percentage of the given class's testing image split that the image classifier correctly predicts. We consequently define our system's classification accuracy to be the image classifier's average prediction accuracy over each class.

\item [NFR2] \textbf{Adaptability}: Adaptability is the ability to change the system to deal with additional application domain concepts \cite{bruegge2004object}. Our system needs to be dynamic enough to accomodate for new small mechanical parts that can be later added even after the system has been deployed.
\end{itemize}

\section{System Models}

\textit{Note: This section includes important system models for the requirements analysis.}

\subsection{Use Case Model}

\textit{Note: This subsection should contain a UML Use Case Diagram including roles and their use cases. You can use colors to indicate priorities. Think about splitting the diagram into multiple ones if you have more than 10 use cases.
\textbf{Important:} Make sure to describe the most important use cases using the use case table template. Also describe the rationale of the use case model, i.e. why you modeled it like you show it in the diagram.}

\subsection{Analysis Object Model}

\textit{Note: This subsection should contain a UML Class Diagram showing the most important objects, attributes, methods and relations of your application domain including taxonomies using specification inheritance (see \cite{bruegge2004object}). Do not insert objects, attributes or methods of the solution domain.
\textbf{Important:} Make sure to describe the analysis object model thoroughly in the text so that readers are able to understand the diagram. Also write about the rationale how and why you modeled the concepts like this.}

\subsection{Dynamic Model}

\textit{Note: UML activity diagrams. \textbf{Important:} Make sure to describe the diagram and its rationale in the text.}


%------- chapter 5 -------

\chapter{System design}

\textit{Note: This chapter follows the System Design Document Template in \cite{bruegge2004object}. 
You describe in this chapter how you map the concepts of the application domain to the solution domain. Some sections are optional, if they do not apply to your problem.
Cite \cite{bruegge2004object} several times in this chapter.}

\section{Overview}

\textit{Note: Provide a brief overview of the software architecture and references to other chapters (e.g. requirements analysis), references to existing systems, constraints impacting the software architecture.}

\section{Design Goals}

\textit{Note: Derive design goals from your nonfunctional requirements, prioritize them (as they might conflict with each other) and describe the rationale of your prioritization. Any trade-offs between design goals (e.g., build vs. buy, memory space vs. response time),
and the rationale behind the specific solution should be described in this section}

\section{Subsystem Decomposition}

\textit{Note: Describe the architecture of your system by decomposing it into subsystems and the services provided by each subsystem. Use UML class diagrams including packages / components for each subsystem.}

\section{Hardware Software Mapping}

\textit{Note: This section describes how the subsystems are mapped onto existing hardware and software components. The description is accompanied by a UML deployment diagram. The existing components are often off-the-shelf components. If the components are distributed on different nodes, the network infrastructure and the protocols are also described.}

\section{Persistent Data Management}

\textit{Note: Optional section that describes how data is saved over the lifetime of the system and which data. Usually this is either done by saving data in structured files or in databases. If this is applicable for the thesis, describe the approach for persisting data here and show a UML class diagram how the entity objects are mapped to persistent storage.
It contains a rationale of the selected storage scheme, file system or database, a description of the selected database and database administration issues.}

\section{Access Control}

\textit{Note: Optional section describing the access control and security issues based on the nonfunctional requirements in the requirements analysis. It also describes the implementation of the access matrix based on capabilities or access control lists, the selection of  authentication mechanisms and the use of encryption algorithms.}

\section{Global Software Control}

\textit{Note: Optional section describing describing the control flow of the system, in particular, whether a monolithic, event-driven control flow or concurrent processes have been selected, how requests are initiated and specific synchronization issues}


\section{Boundary Conditions}

\textit{Note: Optional section describing the use cases how to start up the separate components of the system, how to shut them down, and what to do if a component or the system fails.}





%------- chapter 6 -------

\chapter{Case Study / Evaluation}

\textit{Note: If you did an evaluation / case study, describe it here.}

\section{Design}

\textit{Note: Describe the design / methodology of the evaluation and why you did it like that. E.g. what kind of evaluation have you done (e.g. questionnaire, personal interviews, simulation, quantitative analysis of metrics, what kind of participants, what kind of questions, what was the procedure?}

\section{Objectives}

\textit{Note: Derive concrete objectives / hypotheses for this evaluation from the general ones in the introduction.}

\section{Results}

\textit{Note: Summarize the most interesting results of your evaluation (without interpretation). Additional results can be put into the appendix.}

\section{Findings}

\textit{Note: Interpret the results and conclude interesting findings}

\section{Discussion}

\textit{Note: Discuss the findings in more detail and also review possible disadvantages that you found}

\section{Limitations}

\textit{Note: Describe limitations and threats to validity of your evaluation, e.g. reliability, generalizability, selection bias, researcher bias}



%------- chapter 7 -------

\chapter{Summary}

\textit{Note: This chapter includes the status of your thesis, a conclusion and an outlook about future work.}

\section{Status}

\textit{Note: Describe honestly the achieved goals (e.g. the well implemented and tested use cases) and the open goals here. if you only have achieved goals, you did something wrong in your analysis.}

\subsection{Realized Goals}

\textit{Note: Summarize the achieved goals by repeating the realized requirements or use cases stating how you realized them.}

\subsection{Open Goals}

\textit{Note: Summarize the open goals by repeating the open requirements or use cases and explaining why you were not able to achieve them. \textbf{Important:} It might be suspicious, if you do not have open goals. This usually indicates that you did not thoroughly analyze your problems.}

\section{Conclusion}

\textit{Note: Recap shortly which problem you solved in your thesis and discuss your \textbf{contributions} here.}

\section{Future Work}

\textit{Note: Tell us the next steps  (that you would do if you have more time. be creative, visionary and open-minded here.}



\appendix

\chapter{e.g. Questionnaire}

\textit{Note: If you have large models, additional evaluation data like questionnaires or non summarized results, put them into the appendix.}


\clearpage

\listoffigures
\clearpage

\listoftables
\clearpage

\bibliography{thesis}
\bibliographystyle{alpha}

\end{document}
